\documentclass[11pt, a4paper, leqno]{article}
\usepackage{a4wide}
\usepackage[T1]{fontenc}
\usepackage[utf8]{inputenc}
\usepackage{float, afterpage, rotating, graphicx}
\usepackage{epstopdf}
\usepackage{longtable, booktabs, tabularx}
\usepackage{fancyvrb, moreverb, relsize}
\usepackage{eurosym, calc}
% \usepackage{chngcntr}
\usepackage{amsmath, amssymb, amsfonts, amsthm, bm}
\usepackage{caption}
\usepackage{mdwlist}
\usepackage{xfrac}
\usepackage{setspace}
\usepackage{xcolor}
\usepackage{subcaption}
\usepackage{minibox}
\usepackage{subfig}
% \usepackage{pdf14} % Enable for Manuscriptcentral -- can't handle pdf 1.5
% \usepackage{endfloat} % Enable to move tables / figures to the end. Useful for some submissions.



\usepackage{natbib}
\bibliographystyle{rusnat}




\usepackage[unicode=true]{hyperref}
\hypersetup{
    colorlinks=true,
    linkcolor=black,
    anchorcolor=black,
    citecolor=black,
    filecolor=black,
    menucolor=black,
    runcolor=black,
    urlcolor=black
}


\widowpenalty=10000
\clubpenalty=10000

\setlength{\parskip}{1ex}
\setlength{\parindent}{0ex}
\setstretch{1.5}


\begin{document}

\title{Final Project: Replication of Augenblick & Rabin (2018)\thanks{Max Boehringer, Uni Bonn. Email: \href{mailto:s6mxboeh@uni-bonn.de}{\nolinkurl{s6mxboeh [at] uni-bonn [dot] de}}.}}

\author{Max Boehringer}

\date{
    {\bf Preliminary -- please do not quote}
    \\[1ex]
    \today
}

\maketitle


\begin{abstract}
In their paper "An Experiment on Time Preference and Misprediction
in Unpleasant Tasks" \citet{augenblick2019experiment} investigated the time-inconsistent taste
and future misprediction. Across seven weeks they let a total of 100 participants choose the
number of unpleasant transcription tasks given various wages to complete immediately
and at different future dates. Their main estimate of the present bias parameter  was 0.83,
which could successfully be replicated in table 1.

\end{abstract}
\clearpage

\section{Introduction} % (fold)
\label{sec:introduction}
For the replication I used the python tools estimagic (\citet{Gabler2021}), pytask (\citet{Raabe2020})
and the project template for reproducible economics (\citet{GaudeckerEconProjectTemplates}).
The replication is based on the work of \citet{PozziNunnari}.

\section{Table 1: Primary Structural Estimates}
\label{sec:table1}
\input{tables/table1}
\\
\section{Table 2: Summary statistics for individual structural estimates}
\label{sec:table2}
\input{tables/table2}
\newpage
\section{Plots}
\label{sec:plots}
\begin{figure}[h]
    \centering
    \subfloat[Distribution of individual estimates for \beta]{{\includegraphics[width=5cm]{../../bld/figures/figure1.png} }}
    \qquad
    \subfloat[Distribution of individual estimates for \beta]{{\includegraphics[width=5cm]{../../bld/figures/figure2.png}}}
    \caption{Distribution of individual estimates for \beta and \beta}
\label{fig:fig12}
\end{figure}
\begin{figure}[h]
    \centering
    \subfloat[Distribution of individual estimates for \delta]{{\includegraphics[width=5cm]{../../bld/figures/figure3.png}}}
    \qquad
    \subfloat[Distribution of individual estimates for \gamma]{{\includegraphics[width=5cm]{../../bld/figures/figure4.png}}}
    \caption{Distribution of individual estimates for \delta and \gamma}
\label{fig:fig34}
\end{figure}
\newpage

\bibliography{refs}


\end{document}
